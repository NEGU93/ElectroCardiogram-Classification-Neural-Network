\documentclass[conference]{IEEEtran}
\usepackage{cite}
\usepackage{array}
\usepackage{graphicx}
\usepackage{amsthm}
\usepackage{amsmath}
\usepackage{amssymb}
\usepackage{mathrsfs}
% \usepackage{minted}
\usepackage{listings}
\usepackage{color} 						% red, green, blue, yellow, cyan, magenta, black, white

\usepackage{float} %used to force figures in a position
\hyphenation{op-tical net-works semi-conduc-tor}

\theoremstyle{definition}
\newtheorem{definition}{Definition}[section]

\begin{document}
\title{Clasificación de ECG usando Redes Neuronales}

\author{\IEEEauthorblockN{J. Agust\'{i}n Barrachina}
	\IEEEauthorblockA{Miembro Estudiantil IEEE\\
		Instituto Tecnol\'{o}gico de Buenos Aires\\}}
\maketitle

% \tableofcontents
% \newpage

\begin{abstract}
	Se clasificaron latidos de un segmento de señal de un electrocardiograma según el tipo de arritmia. El algoritmo utilizado fue un perceptrón multicapa usando backpropagation como clasificador. Se utilizaron PCA y SOM para reducir la dimensionalidad de los datos.
\end{abstract}

\section{Introducci\'{o}n}

El trabajo consistió en clasificar los latidos de un segmento de una señal de electrocardiograma de dos canales de 21 horas de duración según el tipo de arritmia.

El objetivo fue crear y entrenar un sistema que pueda reconocer y clasificar los distintos tipos de latidos.

\section{Data Set}

Se utilizaron las grabaciones de "MIT-BIH Long Term Database" \cite{MIT-BIH} disponible en el repositorio PhysioNet \cite{PHYSIONET}.

La grabación incluye anotaciones que identifican la posición y tipo de cada uno de los latidos presentes en la misma.

La grabación se identificará por el número 14172, y el identificador de la base de datos es "ltdb" (\underline{L}ong \underline{T}erm \underline{D}ata\underline{B}ase).
La misma presenta principalmente 4 tipos de latidos:

\begin{itemize}
	\item \textbf{Normales}. Identificados por la letra 'N'.
	\item \textbf{Ventriculares prematuros} Identificados por la letra 'V'.
	\item \textbf{Supraventriculares prematuros} Identificados por la letra 'S'.
	\item \textbf{Nodales prematuros} Identificados por la letra 'J'.
\end{itemize}

Para el tratamiento de los datos se utlizó la librería de python "wfdb" \cite{WFDB}.

\begin{lstlisting}[frame=single]
Error: 	0.290525167175 	Epoch: 1000
Error: 	0.252432576731 	Epoch: 2000
Error: 	0.233333077085 	Epoch: 3000
Error: 	0.226507241997 	Epoch: 4000
Error: 	0.227502496085 	Epoch: 5000
Error: 	0.230580316278 	Epoch: 6000
Error: 	0.233771065888 	Epoch: 7000
Error: 	0.236764878811 	Epoch: 8000
Error: 	0.239612203297 	Epoch: 9000
Error: 	0.242385089641 	Epoch: 10000
\end{lstlisting}

\IEEEpeerreviewmaketitle

\section*{Acknowledgment}

\newpage

\IEEEtriggeratref{8}
\bibliographystyle{IEEEtran}
\bibliography{ref}
\end{document}